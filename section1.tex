\documentclass[11pt]{article}
\usepackage{stat110}
\newif\ifdraft
%\drafttrue % or \draftfalse


\titlespacing\section{0pt}{12pt plus 4pt minus 2pt}{0pt plus 2pt minus 2pt}
\titlespacing\subsection{0pt}{12pt plus 4pt minus 2pt}{0pt plus 2pt minus 2pt}
\titlespacing\subsubsection{0pt}{12pt plus 4pt minus 2pt}{0pt plus 2pt minus 2pt}

\title{Probability and Counting}
\sectionnum{1}

\author{\justin}

\SOLUTION

\begin{document}

\maketitle

\begin{notes}

\section*{The Language of Probability}
Like the English language, the language of probability has its own nouns, verbs, and adjectives.  Confusing these parts of speech will result in "category errors."
\begin{description}
	\item[Experiment] - A process of obtaining outcomes about an uncertain phenomenon.

\item[Sample space] ($\Omega$) - A sample space contains all the possible experimental outcomes that could happen.

\item[Event] -  An event is a certain subset of the sample space.

\item[Probability] - Probability will always be a \textit{value between 0 and 1 inclusive} that expresses how likely a particular event will occur out of all possible outcomes in the sample space.  Moreover, $P(\Omega) = 1$.

    \item[Naive Definition] - If all outcomes are equally likely, the probability of event {A} happening is:
        \[P_{\textrm{naive}}(A) = \frac{\textnormal{number of outcomes favorable to {A}}}{\textnormal{number of outcomes}}\]

\end{description}

\section*{Set Theory}

\begin{description}
        
    \item[Intersection] - Given two events {A} and {B}, $A\cap B$ means {A} \textit{and} {B}. 
    
    \item[Union] - Given two events {A} and {B}, $A\cup B$ means {A} \textit{or} {B}. 
    
    \item[Complement] - Given an event {A}, $A^C$ is called A's complement, and means when "A does \textit{not} occur", meaning everything that's not in {A}.
    
    \item[Subevent] - Given two events {A} and {B}, $A\subseteq B$ means "B includes everything in A".  We can write all valid events A as a subevent of the sample space $\Omega$: $A\subseteq\Omega$.
   \ifdraft
    \item[Multiset] - Given a repeating element $a$ favorable to an event {A}, and an element $b$ favorable to event {B}, we use exponents to simplify the notation of the number of elements in a set $\{a,a,a,a,b,b\}=\{a^4,b^2\}$.
    \fi
        
    \item[De Morgan's Laws] - A useful identity that can make calculations easier by relating unions to intersections. Analogous results hold with more than two sets.
           \begin{align*} 
        ({A} \cup { B})^c = {A^c} \cap { B^c} \\
        ({A} \cap {B})^c = { A^c} \cup { B^c}
           \end{align*}

\begin{minipage}{\linewidth}
            \centering
           %\includegraphics[width=3in]{venn.png}
        \end{minipage}
        \newline
    \item[Principle of Inclusion-Exclusion] - For any events $A_1 ,\dots , A_n$, $$P(\bigcup_{i=1}^n A_i) = \sum_i P(A_i) -\sum_{i<j} P(A_i\cap A_j) + \dots + (-1)^{n+1} P(A_1\cap\dots\cap A_n)$$

    In the small cases that we will usually deal with, this can be written as:
    \begin{align*} 
    P(A \cup B) &= P(A) + P(B) - P(A \cap B) \\
  P(A \cup B \cup C) &= P(A) + P(B) + P(C) \\
&\quad - P(A \cap B) - P(A \cap C) - P(B \cap C) \\
&\quad + P(A \cap B \cap C).
   \end{align*}

My trick using the binomial theorem:
\begin{align} 
    -(\emptyset - A_{i})^n &= -\emptyset^n + {n \choose 1} \emptyset^{n-1}A_{i}^{1} - {n \choose 2} \emptyset^{n-2}A_{i}^2 +\dots + (-1)^{n+1} {n \choose n}A_{i}^n\\
    &={n \choose 1} A_{i}^{1} - {n \choose 2} (A_{i})( A_{i}) +\dots + (-1)^{n+1} {n \choose n}(A_i)(A_i)(A_i)[n times]\\
        &={n \choose 1} A_{i} - {n \choose 2} A_{i}\cap A_j +\dots + (-1)^{n+1} {n \choose n}A_1\cap\dots\cap A_n
\end{align}
\end{description}

\section*{Counting}
\begin{description}
	\item[Multiplication Rule] - If we have $n$ decisions to make and the $j$-th decision has $r_j$ outcomes, then the total number of potential outcomes is $r_1\cdot r_2\cdot\dots\cdot r_{n-1}\cdot r_n$ 
	
	\begin{minipage}{\linewidth}
            \centering
%    \includegraphics[width=4in]{icecream.pdf}
        \end{minipage}

	\item[Binomial Coefficient Formula] -  For $k\leq n$, we have $$ \binom{n}{k} = \frac{n(n-1)\dots(n-k+1)}{k!} = \frac{n!}{k!(n-k)!} $$ 
	$$\binom{n}{k} = \binom{n}{n-k}$$
	\item[Binomial Theorem] - $$(x+y)^n=\sum_{k=0}^n \binom{n}{k}x^k y^{n-k}$$ 
	\item[Factorial] - The number of ways to order $n$ objects is given as $$n! = n\cdot(n-1)\cdot\dots\cdot2\cdot1$$
	\item[Sampling Table] - 
	The sampling table gives the number of possible samples of size $k$ out of a population of size $n$, under various assumptions about how the sample is collected. 
        \begin{table}[H]
        \begin{center}
              \setlength{\extrarowheight}{7pt}
            \begin{tabular}{r|cc}
                 & \textbf{Order Matters} & \textbf{Order Doesn't Matter} \\ \hline
                \textbf{With Replacement} & $\displaystyle n^k$ & $\displaystyle{\binom{n+k-1}{k}}$ \\
                \textbf{Without Replacement} & $\displaystyle\frac{n!}{(n - k)!}$ & $\displaystyle{n \choose k}$
            \end{tabular}
        \end{center}
        \end{table}

\end{description}
\end{notes}

\newpage
\section*{Practice Problems}
\begin{exercise}{Story Proof Practice}
Story proofs are a fundamental and useful way that we will go about proving important results, especially later in the course. To that end, provide story proofs for each of the following results:

\begin{enumerate}

1. Addition suggests combining disjoint cases, while multiplication suggests use of multiplication rule to combine steps of an experiment.

2. Look for patterns similar to the binomial theorem.

    \item
    $$\sum_{k=0}^n {n \choose k } = 2^n$$
    \item 
    $${n \choose k} + {n \choose k-1} = 
    {n+1 \choose k}$$
    \item 
    $$\frac{n!}{(n-k)!k!} = \frac{n \cdot (n-1)
    \dotsm (n-k+1)}{k\cdot (k-1)\dotsm 1}$$
\end{enumerate}
\end{exercise}

\begin{solution}{4}
\begin{enumerate}
    \item Suppose we want to figure out all possible
    numbers we can represent using n bits.  (Bits are binary, taking on a value of either 0 or 1). \\
    
    (RHS) Since each bit can be either 1 or 0, there are $2^n$ possible bits. Because bits take on 2 values of being either a 1 or 0, there would be 
    $2^n$ total possible sequences. \\
    
    (LHS) ${n \choose k}$ represents the number
    of different bit combinations where there are $k$ 1's we can make from $n$ bits.
    ${n \choose 0}$ represents all bits with no 1's,
    ${n \choose 1}$ represents all bits with two 1's,
    and so on. Therefore, $\sum_{k=0}^n {n \choose k}$ represents all possible bits. 
    \item Suppose we have a group of $n$ people and
    a celebrity, and we have to select a committee
    of $k$ people.\\
    
    (RHS) The straightforward way is to just 
    choose $k$ people from the group of $n+1$, 
    giving us ${n+1 \choose k}$ possible 
    committees.\\
    
    (LHS) Suppose we must have the celebrity on the 
    committee. Then there are ${n \choose k-1}$ 
    ways to choose who else will be on the committee.
    Also, suppose that we don't want to have
    the celebrity on the committee. Then there
    are ${n \choose k}$ ways to choose who 
    is on the committee. This gives us a total
    of ${n \choose k-1} + {n \choose k}$ ways
    to pick the committee. 
    \item Recall that these two expressions both
    represent ${n \choose k}$ or the number 
    of ways to choose a committee of size $k$ from
    $n$ people. \\
    
    (LHS) We first permute the $n$ people and then
    select the first $k$ people in the
    permutation. While there $n!$ total permutations,
    we must deal with the overcounting within
    the committee and the people not selected
    for the committee. The $k$ people in the 
    committee can be ordered in $k!$ ways 
    and the people not in the committee can
    be ordered in $(n-k)!$ ways, which indicates that we have overcounted by a factor of $k!(n-k)!$.
    Therefore, we observe that 
        $${n \choose k} = \frac{n!}{k!(n-k)!}$$
    (RHS) Alternatively, we can select the 
    committee directly without permuting the $n$ 
    people. We have $n$ choices for the first
    person on the committee, $n-1$ choices 
    for the second person, and so on. We know
    by the multiplication rule that there
    are $n \cdot (n-1) \dotsm (n-k+1)$ committees
    of size $k$. However, since order doesn't matter
    within the committee, we have overcounted
    by a factor of $k!$. Therefore, we observe
    that
        $${n\choose k} = \frac{n \cdot (n-1) \dotsm (n-k+1)}{k \cdot(k-1)\dots1}$$
\end{enumerate}
\end{solution}
\newpage
\begin{exercise}{Tinder for psets}
So many students are taking Stat110 and so many teaching fellows are hired.  Professor Blitzstein is thinking of creating a new app that is able to pair teaching fellows with students.  But first, he must solve some math problems to pitch to the VCs.

Let Stat110 consist of $n$ students and $n$ TFs so that there are $2n$ people affiliated with Stat110.  An algorithm has already taken the TF's student preferences and the student's TF preferences such that each student is assigned a unique, personal TF.  This algorithm maximizes fulfilled preferences for the class as a whole.  This unique student-personal TF pair will be called a "couple."

At any moment of time, $k$ Stat110 affliates are available to do pset-related activities.  All Stat110 affiliates are available at each given moment of time with equal probability.

Given this information that $k$ Stat110 affiliates are available to do pset-related activities, find the probability that there are exactly $j$ couples doing pset-related activities. 

\end{exercise}

\begin{solution}{2.5}
We know that there are ${2n \choose k}$ total people available for pset activities. Next, we know there must be exactly $j$ couples available, and there are ${n \choose j}$ ways to decide which couples will be on available. Now, it is simply a matter of ensuring that the rest of the $k-2j$ spots are filled with non-couples. \\

We have ${n-j \choose k-2j}$ ways to choose couples from the remaining couples. Then, within each of the $k-2j$ pairs, we only need one partner to be available (and the other must be unavailable), which means there are $2^{k-2j}$ ways to choose which partner in each couple is available. \\

Therefore, the probability that there are exactly $j$ couples available and therefore using the app is 
    $$\frac{{n \choose j} {n-j \choose k-2j} 2^{k-2j}}{{\binom{2n} {k}}}$$
\end{solution}

\begin{exercise}{Poker Probabilities.}
Suppose we have a standard 52-card deck, from which you are dealt five cards. Compute the probability of each of the following hands:
    \begin{enumerate}
        \item A royal flush (getting 10, Jack, Queen, King, and Ace of the same suit).
        \item A flush (all of the cards are of the same suit).
        \item A straight (all five cards are in consecutive order)
        \item A three-of-a-kind (three cards show the same number, and the other two cards do not form a pair)
        \item A two-pair (two cards form a pair and another two cards form a different pair)
    \end{enumerate}
\end{exercise}

\begin{solution}{5}
We begin by noting that there are $\binom{52}{5}$ possible five card hands to start from. 
    \begin{enumerate}
        \item For each of the four suits, there is only one way to obtain a royal flush. Therefore,
        the probability is simply $\frac{4}{{52 \choose 5}} \approx 0.0002$.
        \item To obtain a flush, we must first choose a suit and then choose 5 cards from
        that suit. Therefore, the probability is $\frac{4 {13 \choose 5}}{{52 \choose 5}} \approx
        0.0019$.
        \item To obtain a straight, we first consider what the first card in our straight can be. 
        Since the five cards must be in consecutive order, we have 10 possibilities for the first
        card in our straight (that is, a straight cannot begin with a Jack, Queen, or King). However,
        once we know what the first card in the straight is, the next four cards are determined. All
        that remains is deciding which suit each card will be! Therefore, the probability is
        $\frac{10 \cdot 4^5}{{52 \choose 5}} \approx 0.0039$.
        \item We first decide which value we want our three-of-a-kind to show, which gives us 13 options.
        From there, we have a choice of which 3 of the 4 suits we want it to be. For the next two cards, 
        we require that they show different values, which means there are ${12 \choose 2}$ choices 
        for what they can be, and each of these cards can be any suit. Therefore, the probability is 
        $\frac{13{4\choose 3}{12 \choose 2}4^2}{{52\choose 5}} \approx 0.021$.\\
        
        Alternatively, for the next two cards, we know we have 48 options for the first card and 44
        options for the second card. However, we divide by 2 to deal with the overcounting. 
        \item We first decide which two values we want on each pair, which gives us ${13 \choose 2}$ options. Within each pair, we have the choice of two suits, or ${4 \choose 2}^2$ ways to choose
        the suits of the two pair. The last card must be a different value from either of the pairs, 
        and there are 11 choices with 4 choices of suit. Therefore, the probability is 
        $\frac{{13\choose 2}{4\choose 2}^2 11 \cdot 4}{{52 \choose 5}} \approx 0.047$
    \end{enumerate}
\end{solution}

\begin{exercise}{Brainteaser (credit to Tim Kang)}
Suppose the probability of at least one car passing you at an intersection over the course of twenty minutes is given by 0.9. What is the probability that at least one car passes you over the course of five minutes? Assume that time intervals of the same length have the same probability of observing at least one car. 
\end{exercise}

\begin{solution}{2}
We know that 
    $$P(\text{at least one car in twenty minutes}) = 1 - P(\text{no cars in twenty minutes})$$ 
However, we also know that 
    $$P(\text{no cars in twenty minutes}) = (1-P(\text{at least one car in 
     five minutes}))^4$$
Plugging in 0.9 and solving, we find that the probability that at least one car passes over the course of five minutes is approximately 0.44.      
\end{solution}
\end{document}
